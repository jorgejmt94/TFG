\documentclass[../all.tex]{subfiles}
\begin{document}
%%%%%%%%%%%%%%%%%%%%%%
\section{Introducción}
%%%%%%%%%%%%%%%%%%%%%%
%\newpage
\subsection{Contexto} %marco de trabajo del proyecto + motivación
    
    En la era de la tecnología en la que vivimos cada vez se da más importancia a las redes sociales. Ya no se usan las redes sociales exclusivamente para hablar y compartir cosas con amigos, ahora también lo usan las marcas para hacer campañas publicitarias, los informativos para adelantar noticias y los equipos para informar a sus seguidores. Además de la importancia que de por sí tienen, actualmente, la sociedad informatizada en la que nos hemos adentrado en el último siglo, ha hecho que el interés por el análisis de la opinión y veracidad de la información crezca exponencialmente en estos últimos años.\\
    \newline
    Los seres humanos siempre han querido conocer de inmediato los acontecimientos que suceden a su alrededor. Para ello antiguamente se recurría de los diarios pero solo te decían lo que había sucedido el día anterior, después fue el tiempo de la televisión con sus informativos dos veces al día, a continuación apareció internet y los diarios y cadenas de televisión se aprovecharon de ello para ganarle la partida al otro e informar de los hechos en cuestión de minutos.\\
    \newline
    A partir de esta guerra interna que hay en el mundo del periodismo de informar el primero y tener imágenes de todo el mundo, actualmente, los periodistas, se ayudan de las redes sociales para enterarse de lo que sucede en el mundo. Esto es una gran idea pero, en el mundo de internet, siempre hay gente mintiendo.\\
    \newline
    Dentro de este cambio que han sufrido las redes sociales cada vez es más importante saber en quien se puede confiar y que \textit{posts} hay que ignorar. Quien sube estás noticias son los \textit{trolls}, gente que publica falsos \textit{posts} simplemente para divertirse o porque no saben sobre lo que estan escribiendo o \textit{retweeteando}.\\
    \newline
    De los nombrados \textit{trolls} hay de muchos tipos. Una de las clasificaciones es:
    \begin{enumerate}[resume]
        \item \textbf{El principiante}. Es aquel que se abre un perfil, amparado en el anonimato, y con un número de seguidores insignificante. Generalmente, su trolleo pasará inadvertido y no alcanzará su propósito ni de lejos.
        \item \textbf{El estratega}. Tiene claro el objetivo, contactará con usuarios con identidad digital, buen nivel de penetración, credibilidad y un buen número de seguidores. Hará que ellos se encarguen de propagar su crítica y sin duda pueden conseguir hacer ruido en las redes.
        \item \textbf{El sarcástico}. Te puede sacar de quicio en un momento dado. Le encanta liarte. Su finalidad suele ser la de provocar para que no te olvides de él y por supuesto que no te lo calles. Necesita que cuentes qué hace y que señales quién es a los demás.
        \item \textbf{El sádico}. Su objetivo es disfrutar con tu dolor y humillación y no parará hasta saciar su sed con tu sufrimiento. Es capaz de sacar de contexto cualquier acción que hagas por el mero hecho de ver cómo los demás hacen carnaza con ello. Lo que peor que puedes hacer es contestar a su provocación pues esa será la señal de su ataque.
        \item \textbf{El arrepentido}. Es el que finalmente saca a pasear su conciencia y le sobreponen sus acciones. Pero no te fíes, casi siempre vuelve.
        \item \textbf{El cansino}. Aparece sorpresivamente de la nada metiéndose en una conversación y no para de darte palique hasta límites insospechados. Debes pararle los pies o se convertirá en tu peor pesadilla. Bloquearlo puede ayudar, aunque hay gestores de redes que permiten saltarse el bloqueo.
        \item \textbf{El omnipresente}. No concibe otra forma de estar en redes sociales que no sea trolleando a diestro y siniestro, especialmente a usuarios con muchos seguidores. Su finalidad, en el caso de que no tenga credibilidad, es la del egocentrismo.
        \item \textbf{El lerdo}. Va de listo utilizando un usuario falso y publica el mismo mensaje en su cuenta personal. Le pillas y le hundes, a la par que regalas un buen momento a tus seguidores.
        \item \textbf{El frustrado}. Conocido también como hater. La difamación, el insulto son su sistema de trolleo. Un desencadenante fundamental para ser un hater es el odio. Su frustración y decepción, en algunos casos interna, le hacen imponer su criterio sobre cualquier otro. Es más, harán lo imposible por demostrar que quien no piensa como él es despreciable y humillable. Cualquier cosa que hagas estará mal y será reprochable. Aunque tengas paciencia muy probablemente acabarás buscando un abogado.
        \item \textbf{El oportunista}. Aprovecha el trending topic del momento para insultar a quienes estén relacionados con el tema o para hablar de sus temas recurrentes incluyendo el hashtag sin que éste tenga relación alguna con ellos. Todo por intentar tener un poco de visibilidad.
        \item \textbf{El suplantador}. Simula ser otro usuario para intentar dañar su imagen con mensajes que le perjudican o, si se trata de alguien popular, con el fin de captar muchos seguidores para terminar vendiendo la cuenta.
        \item \textbf{El paródico}. No intenta suplantar a otro usuario, sino parodiarle, unas veces con el fin de burlarse de él de forma más o menos sana y otras con la intención de difamarle.
        \item \textbf{El zombi}. Es una cuenta creada o comprada por un troll con el fin de lanzar mensajes automatizados para atacar a alguien.
        \item \textbf{El vampiro}. Es uno de los más peligrosos. Se alimenta del sufrimiento de sus víctimas y en muchos casos vive obsesionado con ellas, monitorizando absolutamente todas sus actividades dentro y fuera de las redes con la finalidad de lanzar sus ataques. Es un delincuente y cada día hay más casos en los tribunales.
        \item \textbf{El cazador de trolls}. Hay usuarios que terminan vengándose y trollean a sus propios trolls, ya sea desde sus cuentas personales o creando perfiles específicos para la ocasión.
    \end{enumerate}
    
\newpage
\subsection{Motivaciones}
    Las opiniones y sentimientos de las personas siempre han sido una valiosa fuente de información. Por ejemplo, las grandes empresas como Apple o Samsung necesitan conocer el impacto público que producen y mantener controlada al día la opinión pública que será su medidor de éxito de su venta ya que no les vale con vender mucho una vez si no que necesitan vender mucho y muchas veces. Esto les hace recolectar toda la información posible de los clientes ya sea vía mail, redes sociales o cualquier otra forma de saber la opinión de la gente sobre de sus productos. Esto también se aplica a políticos para saber si están enfocando de manera correcta su campaña electoral e incluso las empresas de marketing siempre han tomado la opinión social como el centro de sus actividades. Incluso se ha utilizado esta información como estudios de mercado para venderlos a las empresas y no son más que una recolección de datos.\\
    \newline
    El problema de estos textos que depositamos en nuestras redes sociales es la fiabilidad de la fuente. Cualquier persona que tenga una cuenta puede acceder y comentar sobre el tema que el quiera y ahí es donde aparecen los trolls comentados anteriormente. A veces son comentarios inofensivos que no molestan a nadie como serian comentarios no verificados del estilo de 'Yo cocino mejor que Ferran Adria' donde no hay ninguna maldad sino que se podría considerar como un comentario irónico o gracioso.\\
    \newline
    El motivo principal para la elección de este proyecto ha sido raíz a las mentiras que se llevaron a cavo durante el atentado de Barcelona en las ramblas yo era una de las personas que estaban de vacaciones pero que tenía familia en Barcelona y que la gente con o sin maldad se dedicaba a retwitear todo sin saber ni siquiera si era verdad o se trataba de una fuente fiable y lo único que hacían era provocar pánico, angustia y desconcierto a miles de familias.
    

\newpage
\subsection{Objetivos Trabajo Final de Grado}
    Con el auge de las redes sociales en la última década, los usuarios de redes sociales tales como Twitter, Facebook o Instagram no paran de crecer, y en ellos los usuarios no paran de volcar y compartir opiniones y sentimientos sobre cualquier tema, evidentemente lo que más se comenta es lo que suceda en la actualidad de aquel momento, creando así cantidades inmanejables de datos. El problema de estos datos es el analizarlos y guardarlos de forma estructurada, que sea útil y que nos permita analizar los resultados de forma que éstos nos lleven a conclusiones beneficiosas.\\
    \newline
    Este proyecto se centrará en cazar a los \textit{trolls} de \textit{Tweeter} que  intentan engañar en el mundo de los deportes. Se pretende clasificar \textit{tweets} falsos y verdaderos clasificando estos primero en el deporte al que se vincula el \textit{tweet} (fútbol, baloncesto, balonmano, golf, rugby, etc), según el tono en el que está escrito con el uso de análisis del sentimiento, esto pretende no solo decir si el texto es positivo o negativo sino una clasificación más refinada sobre tipo de sentimiento con el que esta escrito. Finalmente, con estas técnicas en el tweet analizado y tanto anteriores del usuario como tweets con sus hashtags y, habiendo aplicado también una formula propia de fiabilidad, se querrá dar un porcentaje de fiabilidad del tweet.\\
    \newline
    Como ya se ha comentado nos centraremos en la sección de noticias donde hay más mentiras (los deportes) pero, se pretende hacer de tal manera que, solo cambiando los diccionarios de palabras usadas, se puedan aplicar las mismas técnicas y nos permita conocer "a tiempo real" la fiabilidad de los tweets que se leen o que se están volcando en twitter en ese momento. 

\newpage
\subsection{Estructuración de la memoria}
{\color{red} Para el final}
\end{document}
