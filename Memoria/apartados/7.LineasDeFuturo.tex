\documentclass[../all.tex]{subfiles}
\begin{document}
%coste del proyecto
%%%%%%%%%%%%%%%%%%%%%%%%%%%%%%%%%%%%%%%%%
\section{Líneas de futuro} 
%%%%%%%%%%%%%%%%%%%%%%%%%%%%%%%%%%%%%%%%%


Este ha sido un trabajo muy ambicioso dado que quería acaparar un gran volumen de conceptos y no se ha podido llevar a cabo todas las ideas que se tenían puesto que el tiempo para su realización era acotado. Con el tiempo que se ha tenido para su implementación se ha tenido un resultado satisfactorio, pero como todos los proyectos, se empieza con una idea de cómo se quiere hacer y al largo de su codificación se van teniendo más y más ideas de cómo podría ser mejor o en que más ámbitos se podría aplicar con unos simples cambios.\\

Sin ir más lejos, en esta misma universidad en el mes de junio se entregó un trabajo final de máster que de recogía información de Twitter y cuya finalidad era predecir los resultados de los equipos de fútbol en la Premier League. Bien, él sólo cogía la información de las cuentas oficiales de los equipos,  con el trabajo presentado anteriormente se podría reorganizar un poco para que, únicamente ampliando los diccionarios, se busque por todo Twitter quien está hablando de fútbol y después detectar sobre qué equipo hablan ampliando así el dataset del trabajo y como todos sabemos, cómo más grande son los datasets (si los datos son útiles) más probabilidad tenemos de que nuestro algoritmo tenga buenos resultados dado que se entrena con un número mayor de datos.\\

La primera gran restricción de este trabajo fue el tiempo, este factor hizo que se acotara la búsqueda de trolls a solo el mundo de los deportes, pero con más tiempo y personal, lo primero que se podría hacer es aumentar el número de diccionarios con la finalidad de poder abarcar más temas como accidentes o el tiempo e incluso dentro de los deportes separarlos por clubes y jugadores para hacer un análisis aún más exhaustivo. Gracias a que esto ya se sabía desde un principio, se realizó una implementación que permite fácilmente la ampliación de los diccionarios y con lo único que se tendría que pelear es con las hipótesis como que quien habla mucho de fútbol raramente hablará de rugby puesto que son deportes que no se combinan demasiado bien.\\

Otra gran barrera de este trabajo fue el idioma. El idioma con el que trabajan todos los algoritmos ya implementados es el inglés y este trabajo lo hace con el español. Esto implico a no poder usar muchos de los algoritmos previamente implementados en librerías y tener que hacer uso de una implementación propia. Se pretende en un futuro en traducir los diccionarios a otros idiomas y con la codificación propia de los algoritmos, no tendría que haber demasiados problemas en poder usar esas traducciones y así incluso, se podrá resolver el problema de los usuarios que por ejemplo escriben en castellano, pero utilizan expresiones en inglés.

Enlazado con el punto anterior, también se pretende hacer en un futuro un crecimiento en el análisis de los emoticonos que escriben los usuarios y que en muchas ocasiones te pueden dar más información que el texto que ha escrito dado que los emoticonos tienen un alto grado de información.\\

Otra posible ampliación para mejorar el análisis de los algoritmos es la corrección automática de los twits dado que en Twitter se habla con un lenguaje vulgar y se cometen tanto faltas de ortografía como abreviaciones que no existen como sería el caso de tk que significa te quiero.\\

Con el alto grado de estudio de cada usuario, si se ampliaran los temas de los diccionarios, también se podría usar para ayudar al algoritmo de Twitter de sugerir gente para seguirla dado que se conoce sobre qué temas ha escrito y por lo tanto sus intereses.\\

A partir del análisis del sentimiento se podría sugerir a los usuarios de dejar de seguir a gente considerada agresiva, que habla con malos modos o que siempre están enfadados dado que en las redes sociales nadie quiere tener problemas.\\

Este trabajo también nos da la facilidad de recopilar información de los usuarios por temáticas, es decir, se podría usar, añadiendo nuevamente nuevos temas en los diccionarios, para poder hacer extracciones de datos tanto como de fútbol, como del tiempo o de cualquier tema que seamos capaces de crear un diccionario.\\

Como hemos podido ver este trabajo ofrece un abanico muy grande tanto para mejorarlo como para hacer de base en nuevos proyectos como serían sugerir gente a quien seguir, la extracción de información de twitter, avisar a la gente que siguen a gente agresiva o incluso revolucionar el mundo de las apuestas.
\end{document}
