\documentclass[../all.tex]{subfiles}
\begin{document}
%coste del proyecto
%%%%%%%%%%%%%%%%%%%%%%%%%%%%%%%%%%%%%%%%%
\section{Costes del proyecto} 
%%%%%%%%%%%%%%%%%%%%%%%%%%%%%%%%%%%%%%%%%

\subsection{Costes temporales} 
	\begin{center}
		\begin{tabular}{ | m{3cm} | m{4cm}|  } 
			\hline
			\textbf{ } & \textbf{Horas} \\ 
			\hline
			Diseño & 140h\\ 
			\hline
			Creación de \newline \textit{datasets} & 130h\\ 
			\hline
			Implementación & 230h\\ 
			\hline
			Test & 60h\\ 
			\hline
			Memoria & 100h\\ 
			\hline
			\hline
			Total & 660h\\ 
			\hline
		\end{tabular}
	\end{center}
	\begin{figure}[H]
		\centering
		\includegraphics[height=15cm, width=13cm]{imgs/Horas.png}
		\caption{Coste temporal del proyecto.}
	\end{figure} 

\subsection{Costes económicos} 
	
	Este proyecto se ha realizado durante 5 meses a media jornada por un estudiante de ingeniería informática. Esto conlleva a un precio de 4.800€. Gracias a que la persona que ha realizado el proyecto es un estudiante ha tenido el software (\textit{PyCharm}) gratis gracias al programa de estudiantes de \textit{JetBrains}. Las bases de datos se han guardado en un \textit{MongoDB} creado en el mismo ordenador del programador por lo que no se ha tenido que pagar por el almacenamiento de los datos.


\end{document}
