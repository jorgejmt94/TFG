\documentclass[../all.tex]{subfiles}
\begin{document}
%%%%%%%%%%%%%%%%%%%%%%%%%%%%%
\section{Trabajo Relacionado} %Antecedentes
%%%%%%%%%%%%%%%%%%%%%%%%%%%%%
    
    Este trabajo se trata de un proyecto muy ambicioso y como tal abarca un gran abanico de temáticas, por lo tanto, hay mucho trabajo relacionado con partes del proyecto. La gran diferencia es que este proyecto esta en español (la mayoría están en inglés) y que no se centra solo en un concepto.\\
    
    Empezamos con un proyecto que se essta llevando a cabo desde el 2014 hay un proyecto financiado por la UE llamado \textbf{\textit{Pheme}}. El cual es llamado el detector de mentiras de internet. Pheme dice pretender solucionar el cuarto problema del Big Data; la veracidad. Analizará las publicaciones sociales en redes como Twitter y Facebook, las comparará y las contrastará con otras fuentes de información, con el fin de establecer su veracidad y, en el caso de que se clasifique como bulo, señalar la razón (estableciendo si estamos ante una simple especulación, una controversia, un ejercicio deliberado de desinformación, etc).\\

    Luego hay muchos trabajos relacionados respecto a la psicología que hay que aplicar para detectar perfiles falsos en twitter como un articulo que presento el diario abc 
    \footnote{\tiny\url{http://www.abc.es/tecnologia/redes/20140320/abci-faketwitter-saber-twitter-falsos-201403191326.html}} y también sus características.
    \footnote{\tiny\url{http://www.outono.net/elentir/2013/02/02/twitter-5-formas-de-identificar-a-un-troll}}. Incluso hay quien le dio la vuelta a la tortilla y decidio buscar las características de los tweets confiables \footnote{\tiny\url{http://www.josemorenojimenez.com/2012/03/20/caracteristicas-de-un-tweet-confiable/}}.\\

    De trabajos teóricos para detectar noticias falsas hay muchos
    \footnote{\tiny\url{https://www.entrepreneur.com/article/292342}}
    \footnote{\tiny\url{http://www.eltiempo.com/tecnosfera/novedades-tecnologia/como-identificar-noticias-falsas-en-redes-sociales}}
    \footnote{\tiny\url{http://www.abc.es/tecnologia/redes/20131210/abci-como-detectar-noticia-falsa-201312092125.html}} pero no los hay de aplicarlo que se haga automáticamente en una maquina con un programa, el factor humano es clave para ellos. También respecto a los tipos de troll que existen \footnote{\tiny\url{https://blogs.elconfidencial.com/tecnologia/elclubdelalucha/2015-05-09/troll-redes-sociales-ciberacoso_790558/}}
    \footnote{\tiny\url{http://www.elmundo.es/f5/comparte/2017/10/19/59e77fbd22601db82d8b459f.html}} incluso papers donde se hablan de las llamadas \textit{fake news} \footnote{\tiny\url{http://www.kdd.org/exploration_files/19-1-Article2.pdf}}.\\

    Después de todos estos trabajos teóricos también podemos encontrar algunas implementaciones para encontrar bots \footnote{\tiny\url{https://botometer.iuni.iu.edu/\#!/}} e incluso herramientas para encontrar seguidores falsos \footnote{\tiny\url{https://www.genbeta.com/redes-sociales-y-comunidades/7-herramientas-para-detectar-si-tienes-seguidores-falsos}}.\\
    
    Este tema de encontrar trolls y de la preocupación que hay de que la tecnologia no acabe con las personas es una area de interes donde se han introducido grandes empresas como Google con su 
    \textit{Troll detection} \footnote{\tiny\url{https://www.theverge.com/2017/2/23/14713496/google-jigsaw-perspective-software-ai-machine-learning-developers}}, la librería de python de referencia Scickit-Learn \footnote{\tiny\url{http://blog.kaggle.com/2012/09/26/impermium-andreas-blog/}}
    e incluso el gran medio de comunicación internacional BBC habla al respecto
    \footnote{\tiny \url{http://www.bbc.com/news/technology-38181158}}.\\
    
    Por otra parte, esta clasificación de los usuarios en trolls, tambien hay quien,contando que hay trolls agresivos, y no que solo dicen mentiras si no que intentan incordiar a los usuarios o que aplican lo que ahora se conoce como Cyber Bulling, dado esto, se han escrito articulos para la detección de estos usuarios \footnote{\tiny\url{https://academic.oup.com/jigpal/article-abstract/24/1/42/2893010?redirectedFrom=fulltext}}.\\
    

   

\end{document}
