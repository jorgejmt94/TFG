\documentclass[11pt]{article}
\usepackage[utf8]{inputenc}
\usepackage[spanish]{babel}
\usepackage[final]{pdfpages}  % To include pdf files
\usepackage{graphicx}
\usepackage{fancyhdr}
\usepackage{listings}
\usepackage{xcolor}
\usepackage{enumitem}
\usepackage{float}
\usepackage[hidelinks]{hyperref}
\usepackage{url}
\usepackage{macros}
\usepackage{minted}
\usepackage{import}
\usepackage{afterpage}
\usepackage{cite}
\usepackage{makeidx}
\usepackage{pdfpages}
\graphicspath{ {figures/} }
\usepackage{etoolbox,refcount}
\usepackage{multicol}


\usepackage[export]{adjustbox}


\usepackage{array}
\usepackage{subfiles}% para organizacion en ficheros
\usepackage{tikz}%to checkmark
\def\checkmark{\tikz\fill[scale=0.4](0,.35) -- (.25,0) -- (1,.7) -- (.25,.15) -- cycle;}
%para tener 4sub
\usepackage{titlesec}
\setcounter{secnumdepth}{4}
\titleformat{\paragraph}
{\normalfont\normalsize\bfseries}{\theparagraph}{1em}{}
\titlespacing*{\paragraph}
{0pt}{3.25ex plus 1ex minus .2ex}{1.5ex plus .2ex}



\usepackage[a4paper, top=1.3in, bottom=1.3in, headheight=0.5in, voffset=0.3in, headsep=0.5in, footnotesep=0.5in]{geometry}


\renewcommand{\baselinestretch}{1.35}
\selectlanguage{spanish} 
\sloppy

%----------------------------------------
%               INFORMACIÓN
%----------------------------------------

\newcommand
    \subtitle{TFG}

\newcommand
    \codename{Departamento de ingeniería informática}

\author{
    Jorge Melguizo\\
    Tutora: Elisabet Golobardes
    }

\date{Curso académico 2017/18}

\renewcommand
    \title{Trolls Detector: Herramienta para la detección de \textit{trolls} en el ámbito de los deportes}    




%------------------------------------
%               HEADER
%------------------------------------

\pagestyle{fancy}
\fancyhf{}
\lhead{\includegraphics[width=3cm]{logo}}
\rhead{Trolls Detector }
\rfoot{Página \thepage}


%------------------------------------
%               INICIO
%------------------------------------
\begin{document}

%-------------------------------------
%               PORTADA
%-------------------------------------
\includepdf[page=-]{aux/PortadaSalle.pdf} 

\makeatletter
    \begin{titlepage}
    	\centering
    	\includegraphics[width=0.35\textwidth]{logo.JPG}\par\vspace{3cm}
    	{\scshape\LARGE \subtitle \par}
    	\vspace{1cm}
    	\vspace{1.5cm}
    	{\huge\bfseries \title\par}
    	\vspace{2cm}
    	{\Large\itshape \@author \par}
    	\vfill
    	 \textsc{\codename}
    	\vfill
    	{\large \@date \par}
    \end{titlepage}
\makeatother


%------------------------------------
%               PreIndice
%------------------------------------


\newpage{\ } 
\thispagestyle{empty} 


\newpage
\pagenumbering{arabic}
\section*{Palabras clave}
Data Science, Twitter, Trust Tweets, Text Classification, Sentiment Analysis, Fake news, Trolls.
\section*{Abstract}
La población cada vez es más exigente, en el mundo del periodismo esto se resume en inmediatez a la hora  de publicar una noticia. Para ello los periodistas recurren cada vez más a las redes sociales como fuente de información y con la exigencia de ser los primeros en publicar los sucesos, no siempre contrastan la información con fuentes fidedignas. Si lo combinamos con que gran parte de la información que está en las redes sociales suele ser falsa hace que un periodista publique una noticia falsa y provocó que la sociedad se crea una mentira por el hecho de haberlo visto en televisión o leído en algún medio de información.\\

Esta herramienta analizará los perfiles del usuario cuyo tweet se este estudiando su veracidad  y para ello se investigará sobre que temas escribe, de que forma los escribe, a quien hace menciones y retweets, la información de su perfil como si su cuenta es verificada, si siempre publica lo mismo y se comparará con los últimos tweets publicados en Twitter sobre esa temática para dar una aproximación de la veracidad del tweet a tratar.\\

Por lo tanto se pretende dar caza a mentiras en las redes sociales (\textit{liar troll}) pero acortándolo al mundo de los deportes, también detectar aquellos usuarios \textit{bots} o \textit{observer troll} que no suben ninguna publicación y solo tienen twitter para espiar o ampliar los seguidores de otros perfiles y, por último, aquellos usuarios que siempre están furiosos y se podrían clasificar como \textit{hater troll}.  Si se detecta alguno de estos casos se le comunicará al usuario y se le aconsejará dejarle de seguir.\\

Se trata de un problema de Big Data, a tiempo real y dinámico. Es un proyecto de análisis del lenguaje natural con todo lo que ello conlleva.\\


%\section*{Resumen}
%La población cada vez es más exigente, en el mundo del periodismo esto se resume en inmediatez a la hora  de publicar una noticia. Para ello los periodistas recurren cada vez más a las redes sociales como fuente de información y con la exigencia de ser los primeros en publicar los sucesos, no siempre contrastan la información con fuentes fidedignas. Si lo combinamos con que gran parte de la información que está en las redes sociales suele ser falsa hace que un periodista publique una noticia falsa y provocó que la sociedad se crea una mentira por el hecho de haberlo visto en televisión o leído en algún medio de información.
%Este proyecto pretende dar caza a mentiras en las redes sociales pero acortándolo al mundo de los deportes así como también detectar aquellos usuarios \textit{bots} o que siempre estén furiosos.
\thispagestyle{empty} 


\newpage{\ } 
\thispagestyle{empty} 


\newpage
\begin{center}
	\textit{A mis padres Herminia y Jorge y a mis abuelos Herminia y Benedicto por estar siempre a mi lado.}\\
	\textit{A Elisabet Golobardes por haberme dado esta oportunidad y haberme escuchado siempre.}
\end{center}
\thispagestyle{empty} 


\newpage{\ } 
\thispagestyle{empty} 





%------------------------------------
%               ÍNDICE
%------------------------------------
\newpage
\tableofcontents

%\newpage{\ } 
%\thispagestyle{empty} 

\newpage
\listoffigures

\newpage{\ } 
\thispagestyle{empty} 


\newpage
\section*{Acrónimos}
\begin{flushleft}
	SVM: Support Vector Machine.\\
	AVL Tree: Adelson-Velskii y Landis Tree.\\
	XGBoost: Xtreme Gradient Boosting Model.\\
	PLN: Procesamiento de Lenguajes Naturales \\
	GLM: Modelos Lineales Generalizados\\
	NLTK: Natural Language ToolKit.\\
	RFM: Random Forest Model.\\
	OS: Operative System.\\ 
	DB: Data Base.\\
	FC: Fútbol Club.\\
	NB: Naive Bayes.\\

\end{flushleft}
\thispagestyle{empty} 


\newpage{\ } 
\thispagestyle{empty} 

%---------------------------------------
%               CONTENIDO
%---------------------------------------

\newpage
\subfile{apartados/1.Introduccion}

\newpage
\subfile{apartados/2.TrabajoRelacionado}

\newpage
\subfile{apartados/3.Proyecto}

\newpage
\subfile{apartados/4.Experimentacion}

\newpage
\subfile{apartados/5.Costes}

\newpage
\subfile{apartados/6.Conclusiones}

%\newpage
%\subfile{apartados/7.LineasDeFuturo}

\newpage
\subfile{apartados/8.Bibliografia}

\newpage
\subfile{apartados/9.Anexos}


\newpage{\ } 
\thispagestyle{empty} 

%-----------------------------------
%               FINAL
%-----------------------------------
\end{document}
